\chapter{Conclusão}\label{chap6:conclusao}
    Nesse capítulo apresentaremos uma síntese do trabalho realizado, discutiremos a qualidade dos resultados obtidos e os trabalhos futuros.
    
    \section{Síntese e Qualidade dos Resultados}\label{}
    % - Dizer o que foi apresentado no documento
    % - Dizer como o trabalho foi desenvolvido, citando
    % a metodologia
    % - Dizer como identificamos o problema
    % - Dizer o que foi implementado e a
    % importancia disso
    % - Dizer que nossa contribuição foi 
    % deixar o sistema base um pouco mais próximo
    % dos sistemas encontrados na literatura (todos
    % os sistemas encontrados na literatura possuiam
    % CPA implementado)
    % - Dizer que o cálculo do CPA precisa ser revisado,
    % pois o modo como Kuwata equacionou pode não ser
    % aplicável para o nosso sistema
        Nesse trabalho apresentamos a melhoria de um sistema para veículos de superfície não tripulados (do inglês \textit{"Unmanned Surface Vehicle"} - USV) já existente, realizando a implementação do ponto de maior proximidade (do inglês \textit{"Closest Point of Approach"} - CPA), onde buscamos explorar a oportunidade de realizar um trabalho seguindo uma metodologia científica. Ao comparar o sistema desenvolvido por Jurak~\cite{Jurak2020COLREGS} com a literatura (trabalhos realizados por Kuwata~\etal~\cite{Kuwata2014Safe}, Huang~\etal~\cite{Huang2019Generalized} e Song~\etal~\cite{Song2018Two-level}) percebemos que uma das diferenças era a falta do CPA no sistema de Jurak~\cite{Jurak2020COLREGS}, e optamos por atacar esse ponto. 
        
        Os cálculos necessários para a determinação do CPA tiveram como base teórica o trabalho realizado por Kuwata~\etal~\cite{Kuwata2014Safe}. Dado que as posições e velocidades das embarcações eram informações já utilizadas pelo sistema de Jurak~\cite{Jurak2020COLREGS}, fizemos uso das mesmas para realizar o cálculo do CPA. Realizamos os cálculos necessários de forma relativamente simples, porém a 
        alteração realizada no sistema base para faze-lo considerar o CPA demandou grande esforço, pois tal implementação não deveria comprometer o funcionamento atual do sistema base. 
        
        Como indicado pelos resultados dos testes, conseguimos preservar o funcionamento do sistema base e adicionamos a implementação do CPA. Porém, como apresentado na Seção~\ref{subsection_crossing_right_stopped_vessel_cpa}, em uma situação em que não há risco de colisão, a nossa implementação do CPA pode indicar que há risco de colisão. Isso poderia ser resolvido alterando os limiares da Desigualdade~\ref{eq:cpaThreshold}, porém isso demandaria mais tempo de testes, análises e entendimento dos resultados. Dado o curto tempo de implementação do trabalho, optamos por manter os resultados que obtivemos até então como prova que o conceito do CPA está funcional para determinadas situações. Entretanto, não houveram ocorrências em nossos testes para a situação inversa, em que o CPA informaria que não há risco de colisão em uma situação de risco eminente. 
    
    \section{Trabalhos Futuros}
    
        Para continuação deste trabalho é necessário compreender a razão pela qual houve detecção de risco de colisão quando na verdade não havia. Uma das possibilidades citadas na seção anterior é a alteração dos valores a serem atingidos pelo \tcpa e \dcpa para ser detectado o risco de colisão. Porém, pode ser preciso ter um maior entendimento se a equação apresentada por Kuwata~\etal~\cite{Kuwata2014Safe} é aplicável ao sistema desenvolvido por Jurak~\cite{Jurak2020COLREGS}. Por exemplo, no escopo deste trabalho, assumimos que o valor absoluto do \tcpa encontrado deveria ser usado, desconsiderando resultados negativos, visto que inverteria o sentido da velocidade das embarcações. Kuwata~\etal~\cite{Kuwata2014Safe} 
        não faz menção sobre usar o valor absoluto ou não. Também seria preciso entender se o equacionamento apresentado por Kuwata~\etal~\cite{Kuwata2014Safe} é aplicável a curtas distâncias e baixas velocidades, visto que em seus testes o objetivo do USV estava a 1000m de distância e ele se deslocava a uma velocidade de 4m/s. No nosso caso, objetivos para nosso USV eram definidos a pouco mais de 10m de distância da posição inicial com um deslocamento médio de 0,5m/s.